%设置编码类型和格式
\documentclass[UTF8]{ctexart}
%如果要引入图片 则需要引入包
\usepackage{graphicx}
\usepackage{tabularray}
\usepackage{float}

%在正文之前的被称为前言,需要在正文里面输入一个\maketitle才会显示出来
\title{tex入门}
\author{吴羽珩}
\date{\today}

% 正文
\begin{document}
\maketitle
% 一个换行符只会生成一个空格 但是两个换行符就变成换一个段落


% \includegraphics[width=0.5\textwidth]{x} %head是图片文件的名字可以省略类型 图片路径要在这个目录下
%如果想给图片添加标题 那么可以将其嵌套在figure中 位置不对??
\begin{figure}[H]
    \centering%居中
    \includegraphics[width=0.5\textwidth]{x}
    \caption{我的头像}
\end{figure}



% 加粗文字 \textbf{}指令 
% 斜体字 \textit{}
% 下划线标记的文字 \underline{}
段落1:你好1122334455667788999.
\textbf{加粗}
\textit{斜体}
\underline{下划线标记}

段落2:换段落


% 章节 \section{}  最大的是part 其次是 chapter 最后是 section 平时常用section
\section{这是第一个章节}
正文
%创建一个子章节或者二级章节 \subsection{}
\subsection{这是第一个子章节}
正文
\subsubsection{这是一个三级章节}
正文
\section{这是第二个章节}
正文


%列表的创建: begin 和 end 之间的是一个环境 共享相同的文字格式
%对于无序列表的创建 可以使用itemize 环境 列表中的每一个元素都需要以\item开头
\begin{itemize}
    \item 无序列表1
    \item 无需列表2
    
\end{itemize}


%对于有序列表的创建 我们可以使用 
\begin{enumerate}
    \item 有序列表1
    \item 有序列表2

\end{enumerate}

%数学公式 行内公式写在两个美元符号之间 $ $
$(x+y)^2=x^2+y^2+2xy$
%如果想将数学公式单独放在一行
\[
    1+1=2
\]


%表格的用法
%和图片类似 我们也可以设置居中和设置标题 只不过环境的单词不一样 位置不对?
\begin{table}[H]%所有的这些表格 列表都必须加上[H] 
    \centering
    \begin{tabular}{|p{2cm}|c|r|}%c表示居中 l表示左对齐 r表示右对齐 每一列的数据需要 & 隔开 每一行的数据 \\ 分割 |c|c|c| 为表格添加竖直的边框
        %如果想指定每一列的宽度 可以用p{列宽}
            %使用 \hline 来实现添加横线
        \hline
        单元格1 & 单元格2 & 单元格3 \\
        \hline\hline %写两次就是双横线
        单元格4 & 单元格5 & 单元格6 \\
        \hline
    \end{tabular}

    \caption{表格的标题}

\end{table}

\begin{table}[H]
    \centering
    \begin{tabular}{ccc}
    \hline
     & 实际代价 & 摊还代价 \\ \hline
    PUSH & 1 & 3 \\
    POP & 1 & 3 \\
    申请或释放1个位置 & 1 & 0 \\ \hline
    \end{tabular}
\end{table}

i like\footnote[1]{我是你爹} BASIC\@. What about you?

\end{document}